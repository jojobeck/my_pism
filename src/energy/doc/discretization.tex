\documentclass{article}
\usepackage{amsmath}
\usepackage{hyperref}
\usepackage[left=0.5in,right=0.5in,bottom=1in]{geometry}
\usepackage{fancyhdr}
\parindent=0in \parskip=0.5\baselineskip

\newcommand{\R}{\frac{\Delta t\,K^{n}_{k}}{\Delta z^2\,\rho_{i}}}
\newcommand{\mufactor}{\frac{\Delta t}{2\,\Delta z}}
\newcommand{\discretization}{\rho_{i}\,\left(\frac{w^{n}_{k}\,\Delta\left(E^{n+1}_{k}\right)\,\lambda}{2\,\Delta z}+\frac{w^{n}_{k}\,\operatorname{Up}\left(E^{n+1}_{k} , w^{n}_{k}\right)\,\left(1-\lambda\right)}{\Delta z}+\frac{E^{n+1}_{k}-E^{n}_{k}}{\Delta t}\right)-\left(\frac{K^{n}_{k+\frac{1}{2}}\,\mathbf{\delta}_{+}\left(E^{n+1}_{k}\right)}{\Delta z}-\frac{K^{n}_{k-\frac{1}{2}}\,\mathbf{\delta}_{-}\left(E^{n+1}_{k}\right)}{\Delta z}\right)\,\frac{1}{\Delta z}=\Phi^{n}_{k}}
% Values of E on the grid, for the generic equation.
\newcommand{\E}{E^{n+1}}
\newcommand{\El}{E^{n+1}_{k-1}}
\newcommand{\Eu}{E^{n+1}_{k+1}}
% Matrix entries, w > 0.
\newcommand{\Lp}{w^{n}_{k}\,\mu\,\left(\lambda-2\right)-R^{n}_{k-\frac{1}{2}}}
\newcommand{\Dp}{-2\,w^{n}_{k}\,\mu\,\left(\lambda-1\right)+R^{n}_{k+\frac{1}{2}}+R^{n}_{k-\frac{1}{2}}+1}
\newcommand{\Up}{w^{n}_{k}\,\mu\,\lambda-R^{n}_{k+\frac{1}{2}}}
\newcommand{\Bp}{\frac{\Delta t\,\Phi^{n}_{k}}{\rho_{i}}+E^{n}_{k}}
% Matrix entries, w < 0.
\newcommand{\Lm}{-w^{n}_{k}\,\mu\,\lambda-R^{n}_{k-\frac{1}{2}}}
\newcommand{\Dm}{2\,w^{n}_{k}\,\mu\,\left(\lambda-1\right)+R^{n}_{k+\frac{1}{2}}+R^{n}_{k-\frac{1}{2}}+1}
\newcommand{\Um}{-w^{n}_{k}\,\mu\,\left(\lambda-2\right)-R^{n}_{k+\frac{1}{2}}}
\newcommand{\Bm}{\frac{\Delta t\,\Phi^{n}_{k}}{\rho_{i}}+E^{n}_{k}}
% Neumann B. C. at the base.
\newcommand{\neumannb}{\frac{E^{n+1}_{1}-E^{n+1}_{-1}}{2\,\Delta z}=G_{0}}
\newcommand{\Dpb}{-2\,w^{n}_{0}\,\mu\,\left(\lambda-1\right)+R^{n}_{\frac{1}{2}}+R^{n}_{-\frac{1}{2}}+1}
\newcommand{\Upb}{2\,w^{n}_{0}\,\mu\,\left(\lambda-1\right)-R^{n}_{\frac{1}{2}}-R^{n}_{-\frac{1}{2}}}
\newcommand{\Bpb}{2\,G_{0}\,\Delta z\,\left(w^{n}_{0}\,\mu\,\lambda-2\,w^{n}_{0}\,\mu-R^{n}_{-\frac{1}{2}}\right)+\frac{\Phi^{n}_{0}\,\Delta t}{\rho_{i}}+E^{n}_{0}}
\newcommand{\Dmb}{2\,w^{n}_{0}\,\mu\,\left(\lambda-1\right)+R^{n}_{\frac{1}{2}}+R^{n}_{-\frac{1}{2}}+1}
\newcommand{\Umb}{-2\,w^{n}_{0}\,\mu\,\left(\lambda-1\right)-R^{n}_{\frac{1}{2}}-R^{n}_{-\frac{1}{2}}}
\newcommand{\Bmb}{-2\,G_{0}\,\Delta z\,\left(w^{n}_{0}\,\mu\,\lambda+R^{n}_{-\frac{1}{2}}\right)+\frac{\Phi^{n}_{0}\,\Delta t}{\rho_{i}}+E^{n}_{0}}
% Neumann B. C. at the surface.
\newcommand{\neumanns}{\frac{E^{n+1}_{k_{s}+1}-E^{n+1}_{k_{s}-1}}{2\,\Delta z}=G_{k_{s}}}
\newcommand{\Lps}{2\,w^{n}_{k_{s}}\,\mu\,\left(\lambda-1\right)-R^{n}_{k_{s}+\frac{1}{2}}-R^{n}_{k_{s}-\frac{1}{2}}}
\newcommand{\Dps}{-2\,w^{n}_{k_{s}}\,\mu\,\left(\lambda-1\right)+R^{n}_{k_{s}+\frac{1}{2}}+R^{n}_{k_{s}-\frac{1}{2}}+1}
\newcommand{\Bps}{-2\,\Delta z\,G_{k_{s}}\,\left(w^{n}_{k_{s}}\,\mu\,\lambda-R^{n}_{k_{s}+\frac{1}{2}}\right)+\frac{\Delta t\,\Phi^{n}_{k_{s}}}{\rho_{i}}+E^{n}_{k_{s}}}
\newcommand{\Lms}{-2\,w^{n}_{k_{s}}\,\mu\,\left(\lambda-1\right)-R^{n}_{k_{s}+\frac{1}{2}}-R^{n}_{k_{s}-\frac{1}{2}}}
\newcommand{\Dms}{2\,w^{n}_{k_{s}}\,\mu\,\left(\lambda-1\right)+R^{n}_{k_{s}+\frac{1}{2}}+R^{n}_{k_{s}-\frac{1}{2}}+1}
\newcommand{\Bms}{2\,\Delta z\,G_{k_{s}}\,\left(w^{n}_{k_{s}}\,\mu\,\lambda-2\,w^{n}_{k_{s}}\,\mu+R^{n}_{k_{s}+\frac{1}{2}}\right)+\frac{\Delta t\,\Phi^{n}_{k_{s}}}{\rho_{i}}+E^{n}_{k_{s}}}


\newcommand{\diff}[2]{\frac{\partial #1}{\partial #2}}
\newcommand{\wcase}[3]{\begin{cases} #1, & w^{n}_{#3} < 0,\\#2, & w^{n}_{#3} \ge 0.\end{cases}}
\newcommand{\wcaseeq}[4]{#1 &= \wcase{#2}{#3}{#4}}

\begin{document}
\pagestyle{fancy}
\fancyhead[L]{Discretization of the vertical problem for the conservation of energy}
\fancyhead[R]{Constantine Khroulev}

\section{Introduction}
\label{sec:introduction}

These notes dress up \textsc{Maxima}-generated formulas needed to
implement (and check the implementation of) the energy balance code in
PISM, including the implementation of boundary conditions.

\begin{center}
  \begin{tabular}{ll}
    Symbol or formula & Comment \\
    \hline
    $\rho_{i}$ & ice density\\
    $c_{i}$ & specific heat capacity of ice\\
    $k_{i}$ & thermal conductivity of ice\\
    $\phi$ & heat flux at a boundary\\
    $G = \phi / K$ & enthalpy flux at a boundary (see also (\ref{eq:2}))\\
    $\Delta(U_{k}) = U_{k+1} - U_{k-1}$ & centered finite difference\\
    $\delta_{+}(U_{k}) = U_{k+1} - U_{k}$ & forward finite difference\\
    $\delta_{-}(U_{k}) = U_{k} - U_{k-1}$ & backward finite difference\\
    $\operatorname{Up}(U_{k}, \alpha) =
    \begin{cases}
      \delta_{+}(U_{k}), & \alpha < 0,\\
      \delta_{-}(U_{k}), & \alpha \ge 0.
    \end{cases}$ & first order upwinding\\
  \end{tabular}
\end{center}

\section{Energy balance in a column of ice}
\label{sec:energy-in-a-column}

The conservation of energy equation is

\begin{equation}
  \label{eq:1}
  \rho_{i} \cdot \frac{dE}{dt} = -\nabla\cdot\mathbf{q} + Q,
\end{equation}
where
$\mathbf{q}$, the enthalpy flux, is defined by
\begin{align}
  \label{eq:2}
  \mathbf{q} &= -K \nabla E,\\
  K(E) &=
  \begin{cases}
    k_{i}/c_{i}, & E \le E_{s}\quad\text{(cold ice)},\\
    K_{0}, & E > E_{s}\quad\text{(temperate ice)}.\\
  \end{cases}
\end{align}
Here $E_{s}$ is the pressure-dependent enthalpy of the cold-temperate
transition and $Q$ is the source term consisting of the volumetric
strain heating and basal frictional heating.

Note that $K$ depends on $E$, so the system defined by equations
(\ref{eq:1}) and (\ref{eq:2}) with appropriate boundary conditions is
nonlinear. The PISM implementation linearizes it by treating the time
derivative implicitly and using the \emph{previous} (or
initial) distribution of $E$ to evaluate $K(E)$.

PISM uses a shallow approximation of this equation without horizontal
conduction terms. Here I focus on the vertical direction, so
(\ref{eq:1}) becomes
\begin{equation}
  \label{eq:3}
  \rho_{i} \left( \diff{E}{t} + w\,\diff{E}{z} \right) - \diff{}{z}\left( K \diff{E}{z} \right) = \Phi.
\end{equation}
The term $\Phi$ above combines horizontal advection and the source:
\begin{equation}
  \label{eq:4}
    \Phi = Q - \rho_{i} \left( u\,\diff{E}{x} + v\,\diff{E}{y} \right).
\end{equation}

\section{Interior of the column}
\label{sec:interior}

The discretization scheme for equation (\ref{eq:3}) used in PISM is
\begin{equation}
  \label{eq:5}
  \discretization.
\end{equation}
Note that (\ref{eq:5}) uses a convex combination of centered finite
difference and first-order upwinding approximations of vertical
advection with the weight $\lambda$.

For $w \ge 0$ this can be rewritten as
\begin{equation}
  \label{eq:6}
  (\Lp)\, \El + (\Dp)\, \E_{k} + (\Up)\, \Eu = \Bp,
\end{equation}
and for $w < 0$
\begin{equation}
  \label{eq:7}
  (\Lm)\, \El + (\Dm)\, \E_{k} + (\Um)\, \Eu = \Bm.
\end{equation}

Here
\begin{equation}
  R^{n}_{k} = \R,\quad\quad \mu = \mufactor.
\end{equation}

This corresponds to the following lower-diagonal ($L_{k}$), diagonal
($D_{k}$) and upper-diagonal ($U_{k}$) entries in the tridiagonal
matrix corresponding to the discretization (\ref{eq:5}).

\begin{align}
  \wcaseeq{L_{k}}{\Lm}{\Lp}{k}\\
  \notag\\
  \wcaseeq{D_{k}}{\Dm}{\Dp}{k}\\
  \notag\\
  \wcaseeq{U_{k}}{\Um}{\Up}{k}
\end{align}
The right hand side $b$ has elements
\begin{equation}
  \label{eq:8}
  b_{k} = \Bp
\end{equation}
in both cases.

\newcommand{\Rm}{R^{n}_{k-\frac{1}{2}}}
\newcommand{\Rp}{R^{n}_{k+\frac{1}{2}}}
\newcommand{\W}{w^{n}_{k}}
\newcommand{\wstack}[2]{\left\{\begin{matrix}#1\\#2\end{matrix} \right\}}
This can be rewritten as
\begin{align}
  L_{k} &= -\Rm + \mu\W \wstack{-\lambda}{\lambda - 2},\\
  D_{k} &= 1 + \Rm + \Rp + 2\mu\W \wstack{\lambda - 1}{1 - \lambda},\\
  U_{k} &= -\Rp + \mu\W \wstack{2 - \lambda}{\lambda}
\end{align}
with
\begin{equation}
  \label{eq:12}
  \wstack{a}{b} = \wcase{a}{b}{}.
\end{equation}
\section{Neumann B. C.}
\label{sec:neumann-b-c}

In the case of the initial boundary value system consisting of
equation (\ref{eq:1}) combined with
\begin{align}
  \label{eq:9}
  \left.\frac{\partial E}{\partial z}\right|_{z = 0} &= G_{0}(t),\\
  E(z_{\text{surface}}) &= f(t)
\end{align}
the Neumann boundary conditions at the base are implemented by
combining the generic discretization equation (\ref{eq:5}) with the
equation (\ref{eq:10}) approximating (\ref{eq:9}) and eliminating
$E_{-1}$.
\begin{equation}
  \label{eq:10}
  \neumannb.
\end{equation}
This gives
\begin{align}
  \wcaseeq{D_{0}}{\Dmb}{\Dpb}{0}\\
  \notag\\
  \wcaseeq{U_{0}}{\Umb}{\Upb}{0}\\
  \notag\\
  \wcaseeq{b_{0}}{\Bmb}{\Bpb}{0}
\end{align}

\renewcommand{\Rp}{R^{n}_{\frac{1}{2}}}
\renewcommand{\Rm}{R^{n}_{-\frac{1}{2}}}
\renewcommand{\W}{w^{n}_{0}}

\newcommand{\rhs}[1]{E^{n}_{#1} + \frac{\Delta t \Phi^{n}_{#1}}{\rho_{i}}}

Alternatively,
\begin{align}
  D_{0} &= 1 + \Rm + \Rp + 2\mu\W \wstack{\lambda - 1}{1 - \lambda}\\
  U_{0} &= -\Rp -\Rm + 2\mu\W \wstack{1 - \lambda}{\lambda - 1}\\
  b_{0} &= \rhs{0} + 2G_{0}\Delta z \left( -\Rm + \mu\W\wstack{-\lambda}{\lambda - 2} \right).
\end{align}

When assembling the basal equation we approximate $R_{-\frac{1}{2}}$ by $R_{0}$.

\newcommand{\ks}{k_{s}}

Similarly, when the Neumann B.~C. is imposed at the surface ($k =
\ks$), we combine (\ref{eq:5}) with (\ref{eq:11}) and eliminate $E_{\ks+1}$.
\begin{equation}
  \label{eq:11}
  \neumanns.
\end{equation}
This gives
\begin{align}
  \wcaseeq{L_{\ks}}{\Lms}{\Lps}{\ks}\\
  \notag\\
  \wcaseeq{D_{\ks}}{\Dms}{\Dps}{\ks}\\
  \notag\\
  \wcaseeq{b_{\ks}}{\Bms}{\Bps}{\ks}
\end{align}

\renewcommand{\Rp}{R^{n}_{\ks+\frac{1}{2}}}
\renewcommand{\Rm}{R^{n}_{\ks-\frac{1}{2}}}
\renewcommand{\W}{w^{n}_{\ks}}

Alternatively,
\begin{align}
  L_{\ks} &= -\Rp -\Rm + 2\mu\W \wstack{1 - \lambda}{\lambda - 1}\\
  D_{\ks} &= 1 + \Rm + \Rp + 2\mu\W \wstack{\lambda - 1}{1 - \lambda}\\
  b_{\ks} &= \rhs{\ks} + 2G_{\ks}\Delta z\left( \Rp + \mu\W\wstack{\lambda-2}{-\lambda} \right)
\end{align}

When assembling the surface equation we approximate $R_{\ks + \frac{1}{2}}$ by $R_{\ks}$.

\end{document}
